\documentclass{beamer}
\usepackage{lmodern}
\usepackage{graphicx}
\usetheme{Warsaw}
\setbeamertemplate{navigation symbols}{}
\setbeamertemplate{frametitle}{}

\newcommand{\includeGraph}[2]{
	\begin{center}
	\scalebox{#1}{
	%	\nonstopmode
		\input{images/#2.tex}
	%	\errorstopmode
	}
	\end{center}
}
\newcommand{\includePicture}[2]{
	\begin{center}
	\includegraphics[width=#1\textwidth]{images/#2}
	\end{center}
}

\title{Simulation of a many-particle system using space partitioning}
\author{Roald Frederickx \and Kasper Meerts}
\date{10 November 2010}
\begin{document}

\begin{frame}
\titlepage
\end{frame}

\section{Inleiding}
\subsection{Inhoud}
\begin{frame}
\tableofcontents[hideallsubsections]
\end{frame}

\subsection{Probleemstelling}
\begin{frame}
Veel fysische systemen te modelleren door interagerende deeltjes

Bijvoorbeeld
\begin{itemize}
\item Ideaal gas
\item Elektronen in metaal
\item Diffusie
\item Warmtegeleiding
\item Adsorptie
\end{itemize}

Enkel korte-afstand interactie (``botsen'')!
\end{frame}

\subsection{Na\"ief}
\begin{frame}
\begin{itemize}
\item Elk paar apart bekijken
\item $n(n-1)/2$ paren $\Rightarrow O(n^2)$
\item Veel overbodig werk
\end{itemize}
\includeGraph{0.6}{quadraticComplexity}
\end{frame}

\subsection{Space partitioning}
\begin{frame}
\begin{itemize}
\item Ruimte onderverdelen in ``dozen''
\item $n$ deeltjes
\item $x$ deeltjes per doos
\item $n/x$ dozen
\item Complexiteit $O(n/x \cdot x^2) = O(nx) = O(n)$
\end{itemize}
\begin{center}
\includegraphics[width=0.4\textwidth]{images/grid.pdf}
\end{center}
\end{frame}

\section{Implementatie}
\subsection{Algemeen}
\begin{frame}
\begin{itemize}
\item Programmeertaal: C
\item Harde bollen
\item Elastische botsingen
	\begin{itemize}
	\item A posteriori
	\item Backtracking
	\end{itemize}
\item ``Doos'' = lijst
\end{itemize}
Testen op botsingen
\begin{itemize}
\item Binnen eigen doos
\item Buurdozen
\end{itemize}
\end{frame}

\subsection{Opvullen wereld}
\begin{frame}
\begin{itemize}
\item Genereer willekeurige positie
\item Zolang botsing: probeer opnieuw
\end{itemize}

Volumefractie gestapelde bollen:
\begin{itemize}
\item Maximaal $\approx74\%$ op een regelmatig rooster
\item Willekeurig $\approx63\%$ mits ``schudden''
\item Ons algoritme $\approx52\%$
\end{itemize}
\begin{center}
\scalebox{0.45}{
	\input{images/fillJcurve.tex}
}
\includegraphics[width=0.4\textwidth]{images/maxDensity.png}
\end{center}
\end{frame}

\section{Performance}
\subsection{Effect van partitioning}
\begin{frame}
\begin{columns}
\begin{column}{0.85\textwidth}
\includeGraph{0.75}{fixedPartnum}
\end{column}
\begin{column}{0.15\textwidth}
10\,000 deeltjes
\end{column}
\end{columns}
\end{frame}

\subsection{Ideaal aantal dozen}
\begin{frame}
\includeGraph{0.8}{idealNboxR0p5}
\end{frame}

\begin{frame}
\includeGraph{0.8}{idealNboxR0p1-1M}
\end{frame}

\subsection{Complexiteit}
\begin{frame}
\includeGraph{0.8}{linearComplexityR0p5}
\end{frame}

\begin{frame}
\includeGraph{0.8}{linearComplexityR0p1-1M}
\end{frame}

\subsection{Conclusie}
\begin{frame}
\begin{itemize}
\item $\sim$\,10 dozen per deeltje
\item $O(n^2) \rightarrow O(n)$
\item 5 jaar $\rightarrow$ 1 seconde
\end{itemize}
\end{frame}

\section{Toepassingen}
\subsection{Maxwell-Boltzmann}
\begin{frame}
\begin{columns}
\begin{column}{0.7\textwidth}
\includePicture{1.00}{maxwellRender.png}
\end{column}
\begin{column}{0.3\textwidth}
8\,000 deeltjes
\end{column}
\end{columns}
\end{frame}

\begin{frame}
\includeGraph{0.8}{relax}
\end{frame}

\begin{frame}
\includeGraph{0.8}{maxwell}
\end{frame}

\subsection{Brownse beweging}
\begin{frame}
\begin{columns}
\begin{column}{0.7\textwidth}
\includePicture{1.00}{brownRender.png}
\end{column}
\begin{column}{0.3\textwidth}
10\,000 deeltjes

1 groot deeltje
\end{column}
\end{columns}
\end{frame}

\begin{frame}
\includeGraph{0.6}{brown}
\end{frame}

\subsection{Andere toepassingen}
\begin{frame}
\begin{itemize}
\item Van der Waals correctie
\item Diffusie
\item Warmte-geleiding
\item Klassiek adsorptiemodel
\end{itemize}
\end{frame}

\section{Conclusie}
\begin{frame}
Space partitioning:
\begin{itemize}
\item Korte-afstand interacties
\item $O(n^2) \rightarrow O(n)$
\item Zonder verlies van correctheid
\end{itemize}
\end{frame}

\end{document}
