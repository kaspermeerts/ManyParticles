\documentclass{beamer}
\usepackage{lmodern}
\usepackage{graphicx}
\usetheme{Berkeley}
\setbeamertemplate{navigation symbols}{}

\newcommand{\scale}{0.5}
\newcommand{\figscale}[2]{
	\begin{center}
	\scalebox{#1}{
	%	\nonstopmode
		\input{images/#2.tex}
	%	\errorstopmode
	}
	\end{center}
}
\newcommand{\items}[1]{\begin{itemize} #1 \end{itemize}}

\title{Simulation of a many-particle system using space partitioning}
\author{Roald Frederickx \and Kasper Meerts}
\date{10 November 2010}
\begin{document}

\begin{frame}
\titlepage
\end{frame}

\begin{frame}
\tableofcontents
\end{frame}

\section{Inleiding}
\begin{frame}
\frametitle{Relevantie}
Veel fysische systemen te modelleren door interagerende deeltjes

Bijvoorbeeld
\begin{itemize}
\item Ideaal gas
\item Elektronen in metaal
\item Maxwell-Boltzmann verdeling fi ffi fl
\item Adsorptie
\item Diffusie
\item Warmtegeleiding
\end{itemize}

Enkel korte-afstand interactie!
\end{frame}

\begin{frame}
\frametitle{Simulatie --- Na\"ief}
\begin{itemize}
\item Elk paar apart bekijken
\item $O(n^2)$
\item Veel overbodig werk
\end{itemize}
\figscale{0.6}{quadraticComplexity}
\end{frame}

\begin{frame}
\frametitle{Oplossing: Space partitioning}
\begin{itemize}
\item Ruimte onderverdelen in ``dozen''
\item $n$ deeltjes
\item $x$ deeltjes per doos
\item $n/x$ dozen
\item Complexiteit $O(n/x \cdot x^2) = O(nx) = O(n)$
\end{itemize}
\begin{center}
\includegraphics[width=0.4\textwidth]{images/grid.pdf}
\end{center}
\end{frame}

\section{Implementatie}
\begin{frame}
\frametitle{Implementatie}
\begin{itemize}
\item Programmeertaal: C
\item Harde bollen
\item Elastische botsingen
	\begin{itemize}
	\item A posteriori
	\item Backtracking
	\end{itemize}
\item ``Doos'' = lijst
\end{itemize}
Testen op botsingen
\begin{itemize}
\item Binnen eigen doos
\item Buurdozen
\end{itemize}
\end{frame}

\begin{frame}
\frametitle{Wereld vullen}
\begin{itemize}
\item Genereer willekeurige positie
\item Zolang botsing: probeer opnieuw
\end{itemize}

Volumefractie gestapelde bollen:
\begin{itemize}
\item Maximaal $\approx74\%$ op een regelmatig rooster
\item Willekeurig $\approx63\%$ mits ``schudden''
\item Ons algoritme $\approx52\%$
\end{itemize}
\begin{center}
\scalebox{0.45}{
	\input{images/fillJcurve.tex}
}
\includegraphics[width=0.4\textwidth]{images/maxDensity.png}
\end{center}
\end{frame}

\section{Performance}
\begin{frame}
\frametitle{Performance}
\figscale{0.8}{fixedPartnum}
\end{frame}

\begin{frame}
\frametitle{Performance}
\figscale{0.8}{idealNboxR0p5}
\end{frame}

\begin{frame}
\frametitle{Performance}
\figscale{0.8}{idealNboxR0p1-1M}
\end{frame}

\begin{frame}
\frametitle{Performance}
\figscale{0.8}{linearComplexityR0p5}
\end{frame}

\begin{frame}
\frametitle{Performance}
\figscale{0.8}{linearComplexityR0p1-1M}
\end{frame}

\begin{frame}
\frametitle{Performance conclusie}
\begin{itemize}
\item $\sim$\,10 dozen per deeltje
\item $O(n^2) \rightarrow O(n)$
\item 5 jaar $\rightarrow$ 1 seconde
\end{itemize}
\end{frame}

\section{Toepassingen}
\subsection{Maxwell-Boltzmann}
\begin{frame}

\end{frame}
\end{document}
