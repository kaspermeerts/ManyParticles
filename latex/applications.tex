\section{Applications}

To validate the correctness of the software, we have tested some important 
physical phenomena that are described by statistical mechanics. We studied 
the Maxwell-Boltzmann distribution and Brownian motion. Other interesting 
phenomena include adsorbtion models, the Van der Waals corrections to the 
ideal gas law, diffusion, heat conduction and capacity, phase transitions, 
thin film phenomena, etc...

\subsection{Maxwell-Boltzmann distribution}
The Maxwell-Boltzmann speed distribution describes the probability density of 
the speed of a gas particle. The necessary temperature parameter was calculated 
from our simulation using the equipartition theorem.
$$
E_{\textrm{avg, kin}} = \left< \frac{mv^2}{2} \right> = \frac{3k_BT}{2}
$$
With $m$ the mass of each particle and $k_B$ the Boltzmann constant. Both were 
set to unity in our simulation. Thus
$$
T = \frac{\left< v^2 \right>}{3}
$$
The well known Maxwell-Boltzmann distribution reads \cite{maxwellDist}
$$
f(v)\dif v = \left( \frac{\beta m}{2 \pi} \right)^{3/2} 4 \pi v^2
\exp\left( -\beta \frac{mv^2}{2}\right) \dif v =
\sqrt{ \frac{2}{\pi T^3} } v^2 \exp \left(-\frac{v^2}{2T}\right) \dif v
$$
with T given in function of the expectation of the squared velocity above.

The velocity of our particles got initialized as a vector of fixed length 
1, pointing in a random direction. Thus, the inital speed distribution has 
the shape of a delta distribution around the speed 1. Through collisions 
between the particles, the particles reach thermal equilibrium. While doing 
so, the sharp peak in the speed distribution relaxes to approximate the 
Maxwell-Boltzmann distribution.

\figOctave{maxwell}{Speed distribution of a simulation and the expected 
theoretical distribution for a temperature of 1/3}

The average of three  resulting simulated distributions is shown in figure 
\ref{maxwell} together with the theoretical result. The parameters used 
were: a worldsize of $60 \times 60 \times 60$, 8000 particles with radius 
0.2 and a relaxation time of 15\,000 iterations with timestep 0.05.

Note that the density in our example was about 0.1\% (the fraction of total 
particle volume with respect to the world size). This roughly approximates 
the density of gasses at room temperature. The same result can be found 
with more dense packings, and with less dense (but this involves a lot more 
iterations, as collisions will be very rare).

The temperature of the theoretical curve was obtained from the formula 
above, using that the temperature at the beginning\footnote{Note that it's 
not quite appropriate to talk about a temperature in this case. We aren't 
in thermal equilibrium yet and there is no entropy to calculate $\partial E 
/ \partial S$. Nevertheless, using the equipartitiontheorem to define the 
temperature works just fine.}
was given by $\left<v^2\right>/3$, in our case 1/3.

The results are clearly in perfect agreement with the predicted result.


\subsection{Brownian motion}
With a small change to the software, a single massive, huge particle was 
introduced in the system. Because of its relative size, space partitioning 
could not be used for this new particle and each smaller particle had to be 
checked for interactions with the huge particle separately. This of course 
did not change the linear complexity of the algorithm and had a negligible 
effect on performance.

We placed a particle of radius $10$ and mass 1 in the center of a $320 
\times 320 \times 320$ world without initial velocity. The rest of the 
world got filled randomly with 10\,000 particles of radius 0.5 and of the 
same mass. The velocities of these particles has fixed length of 1 and
point in random directions.

The trajectory of the large particle during four runs of the simulation is 
plotted in figure \ref{brown}. Brownian motion is discernible in the path.

\figOctave{brown}{Four trajectories of Brownian motion starting from the 
center dot. (worldsize $320 \times 320 \times 320$, 10\,000 particles of 
radius 0.5, radius of large particle: 10, mass: 1)}


\subsection{Ideal gas law and Van der Waals corrections}

If every collision with the wall is noted and the change in momentum is 
recorded, an expression for the pressure on the wall can be obtained. Using 
the equipartition theorem to retrieve the temperature, it is possible to 
simulate the ideal gas law with the additional Van der Waals correction for 
particles with finite volume.


\subsection{Heat conduction and specific heat}

Heat conduction is another phenomenon that can be studied. No results are shown 
in this article but we propose a method how one can implement this using the 
software.

The first proposal models heat conduction in a gas. We propose to give two 
opposite walls a different temperature. When a particle collides with a 
wall, the normal component with respect to the wall gets mixed randomly 
with the momentum it would have if it had the same temperature as the wall.  

This is enough to model heat transfer from one wall to another. With a 
sufficiently dense gas, and a sufficient distance between the walls, it is 
even possible to graph a temperature profile by considering the average 
energy of the particles in function of their distance to the walls.

A second proposal models heat conduction by electrons in a metal. The 
setting is the same as the previous example, but with an additional regular 
grid of stationary particles that fills the world. These particles can 
exchange kinetic energy with the moving particles after a collision. The 
stationary particle stores its kinetic energy in a vibration mode. This 
gives an idea of the temperature distribution of the system and doesn't 
require a very dense gas.

By varying the temperature of the walls, the specific heat can be 
determined. The vibration modes of the last probosition could even be 
linked to phonons in the Debye model of a solid to model their contribution 
to the specific heat.


\subsection{Langmuir adsorbtion model}

A simple surface adsorbtion model can be implemented. We propose that particles 
that make contact with a wall can be adsorbed if their momentum normal to the 
wall is small enough. Once adsorbed, their lost kinetic energy is stored in a 
vibration mode and the particles can move freely in two dimensions on the 
wall.  The particle can be desorped if, through interactions with other 
particles, it can gain enough kinetic energy. This second particle can be a 
particle that is itself bound to the wall and moving over its surface, or a 
particle from the gas that collides with bound particle.

The proportion of adsorbed particles given by the Langmuir model is given 
by\cite{adsorbtionModel}
$$
\frac{N_\mathrm{ads}}{N_\mathrm{gas}} =
\frac{A \lambda_\mathrm{th}}{V}e^{\epsilon/kT}
$$
where $N_\mathrm{ads}$ is the number of adsorbed particles, $N_\mathrm{gas} 
$ the number of unbound particles, $\lambda_{\textrm{th}}$ the thermal de 
Broglie wavelength $\lambda_{\mathrm{th}}^2 = h^2 / 2 \pi m k T$, $A$ the 
surface area and $\epsilon$ the binding energy of the wall.

Note the occurence of the constant op Planck in the expression for the 
thermal wavelength. It would be interesting to see if our proposed simple 
(and fully classical) model could be used to estimate this fundamental 
constant of quantum mechanics.


\section{Conclusions}
The simulations of the Maxwell-Boltzmann distribution and Brownian motion 
show that our implementation gives the correct results. The proposed models
for the ideal gas law with a Van der Waals correction, the heat conduction 
and heat capacity and the proposed classical model of adsorbtion are 
interesting subjcts for further study based on the given performant 
framework of colliding spheres with space partitioning.
