\section{Implementation}

\subsection{Programming language}
For this project, performance was a prime goal and thus the only contemplated 
programming languages were C and C++.  Originally, we decided to implement the 
project in C++. It seemed to us that the clear distinction between the system, 
a partition and a particle lent itself well to an object-oriented programming 
language. Another advantage was the Standard Template Library (STL) which 
provided us with an efficient doubly linked list. Lastly, overloading the 
addition and multilpication operators allowed for a natural expression of 
vector operations.

The performance critical nature of the problem, on the other hand, did not 
lend itself too well to the object-oriented programming paradigm. The 
principles of information hiding and encapsulation added a considerable 
overhead to the code. Lastly, we have a lot more experience writing C when 
compared to C++.  Given the time constraints, it was more natural to write 
in a language we knew better.

For these reasons, we decided to switch to C.  This conversion only took an 
hour and the imperative paradigm seemed more fit to develop a performant 
program in a reasonable amount of time.

\subsection{Data structures}

The project needs a lot of manipulation of vector quantities, like position 
and velocity. For this we use the following straightforward C structure

\begin{lstlisting}
struct Vec3 {
	float x, y, z;
};
\end{lstlisting}

A particle is represented by its position and velocity. The particles needs 
to be contained in a box so we store them in a circular doubly-linked list.  
This enables us to easily iterate over all the particles in a box.
\begin{lstlisting}
struct Particle {
	Vec3 pos;
	Vec3 vel;
	struct Particle *prev, *next;
};
\end{lstlisting}
In this case, a box is little more than a pointer to any particle it contains.  
For various reasons, we keep track of the amount of particles in each box.
\begin{lstlisting}
struct Box {
	Particle *p;
	int n;
};
\end{lstlisting}

\subsection{Algorithms}

To set up the initial configuration, the system is filled with the set amount 
of particles. A particle is placed in a random location with a random velocity 
from an isotropic distribution. The program then checks for collisions and if 
necessary, tries another location until it has found an empty spot.

Each iteration, the simulation is advanced one timestep. This step consists of 
first checking for collisions with other particles, then the particles are 
checked for collisions with the walls and lastly every particle is moved in a 
straight line according to its velocity. Any particle that moves over the 
boundary of a partition is transfered to its new box.

To find out whether a particle intersects another particle, we test it against 
every other particle in the same box. As this is the common case, this step is 
performed first. If this search yields nothing, we check against particle in 
neighbouring boxes. Note that it is important that the box size is at least two 
times the radius of each particle for we could miss some collisions otherwise. 
A possible, unimplemented performance improvement would be only checking the 
boxes that intersect the particle. 

The law of conversation of momentum allows to make another small performance 
improvement. The equations for the elastic collision are solved for one of the 
particles. Instead of doing the same for the other particle, the change in 
momentum of the first particle is subtracted from the second particle.

\begin{lstlisting}
function stepWorld()
	for each box:
		for each particle in the box:
			if (the particle collides)
				handleCollision();
	
	for each box at the boundary:
		for each particle in the box:
			if (the particle collides with the wall)
				bounceParticle();
	
	for each particle:
		advanceParticle();
		if (particle is not in original box)
			transferParticle();
\end{lstlisting}

\subsection{Interface}
The parameters for the system are given to the program by commandline 
arguments. Methods to benchmark the performance of the program are also built 
in the software. The scripts used to gather the data are in the same repository 
as the code. Care was taken to only measure the time the processor spent 
calculating and not the time the program took to finish, which is very 
sensitive to external conditions.

At the start of the program the world is filled with a given amount of 
particles. This overhead is not included in the benchmark results. Then, either 
the system is rendered graphically for debugging purposes, or it loops a set 
amount of iterations.  Optionally, the positions and velocities of each 
particle are printed. This way, we can plot a distribution of the speed or 
demonstrate Brownian motion, the results of which are explained in the next 
section.


