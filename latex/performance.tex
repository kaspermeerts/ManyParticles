% vim: ft=tex

\section{Performance}
\subsection{Testing platform}

All tests were run on a laptop with an Intel Core2 Duo P8700 processor and 
4\,GB of dual channel DDR2 800\,MHz RAM running Gentoo Linux on a 2.6.33 
kernel with the kernel timer ticking at 1000\,Hz. All CPU time measurements 
were accurate to 1\,ms.

The code was compiled with version 4.4.5 of the GNU C Compiler using the 
-O3 optimization flag and building for the native architecture. Debugging  
options were omitted for the benchmark runs.


\subsection{Filling the world}
Our algorithm initializes the world by trying to insert a particle in a 
random position. If this particle collides with an existing particle, a new 
random location is tried, otherwise the particle is added.

Of course, there exists a limit to the amount of spheres that can be packed in 
a container. If the total volume of the spheres exceeds the available volume, 
it is of course impossible to place them in this container.  Moreover, even if 
the total volume of the spheres is less than the available volume, there is no 
guarantee that the spheres will fit. Because of the round shape of the spheres, 
there will always be some space that goes to waste.

\figOctave{fillJcurve}{Fraction of failed particle placements w.r.t.  
packing density. Note the logarithmic scale on the vertical axis.}

The problem of trying to fit as many spheres as possible in a fixed 
container is known as \emph{sphere packing}. It is 
proven\cite{idealPacking}
that the most ideal sphere packing has a volume fraction of $\pi / \sqrt{18}
\approx 0.74$. This happens when the spheres are placed on a regular grid
called  ``face centered cubic''. Our implementation, however, fills the 
world in a random, irregular way. This is a closely related problem known 
as \emph{random close packing}. Another theorem\cite{randomPacking} states that 
the packing in such a random way cannot exceed approximately 63\%. This is the 
absolute maximum, assuming the randomly placed particles are ``shaken'' as much 
as possible to let them settle down and reach the lowest possible volume.

Our algorithm did no such ``shaking'', and could only fill the world up to a 
density of 52\%. A graph of the failed particle placements w.r.t. the packing 
density is given in figure \ref{fillJcurve}. Notice the asymptotical behaviour 
near the 52\% density. Above this density, the algorithm will loop forever, 
trying to insert a particle in a random position, but it will never succeed in 
doing so.

A render of the maximum density that our algorithm could achieve is shown 
in figure \ref{maxDensity}.

\fig{width=0.4\textwidth}{maxDensity}{Render of a world with the maximum 
amount of particles.}


\subsection{Collisions: naive approach}
The performance of a naive many-particle system with simple collisions was 
evaluated and the results are shown in figure \ref{quadraticComplexity}.  This 
naive approach was simulated by using just one box for the entire world. The 
quadratic complexity of the problem in function of the number of particles is 
clearly visible.

\figOctave{quadraticComplexity}{Quadratic complexity with one box.  
Worldsize $50 \times 50 \times 50$, radius $0.5$.}



\subsection{Space partitioning}
To ameliorate the above result, the ideal number of boxes needs to be 
determined.  The performance of a typical system in function of the number of 
partitions is given in figure \ref{fixedPartnum}. The performance peaks around 
the ideal number.  We have determined the optimum number of boxes for a large 
amount of particles and graphed it in figure \ref{idealNboxR0p5}.

\figOctave{fixedPartnum}{Effect of number of boxes on performance, for a 
fixed number of particles. Worldsize $50 \times 50 \times 50$, radius $0.1$, 
10\,000 particles.}


In that test, each particle has a radius of $0.5$ and the worldsize was 
kept constant at $50 \times 50 \times 50$. The number of particles ranged 
from 0 to 10\,000. With the maximum number of particles, the world was 
filled for about 4\%, expressed as the fraction of volume of all particles 
with respect to the volume of the world.

From the clear linear trend in this plot, it follows that --- for our 
implementation  --- the ideal number of boxes is roughly ten times the 
number of particles. This means that, on average, there will only be one 
particle in every tenth box. Thus we can conclude that detecting a 
collision is \emph{much} harder than dealing with the overhead of iterating 
over the boxes.

Another interesting feature of this curve is the anomaly found at around 1\,000 
to 2\,000 particles. Here, the ideal number of boxes is constant and takes the 
value 15\,625, or $25^3$. Note that this means that every box has a size of 
exactly $2 \times 2 \times 2$. We believe that the preference for this amount 
of boxes comes from the fact that a computer can easily do arithmetic with 
numbers that are powers of two. Thus this anomaly has more to do with the 
hardware running the algorithm than the algorithm itself. Note that there also 
seems to be a plateau around 8\,000 particles and more. This occurs at 125\,000 
boxes, or exactly $50^3$. Each box now has size $1 \times 1 \times 1$, and 
calculations (divisions, multiplications) with the number 1 can be done quite 
efficiently indeed.

Recall that for the maximum number of particles chosen, only 4\% of the 
available space is filled. However, we could not extend the graph to the 
right with more particles, since this would mean using boxes with a width 
smaller than 1. Given that the chosen radius of a particle was $0.5$, this 
would mean that two particles could collide that are not in adjacent boxes.  
This collision would not get detected and the results would be useless.

The only way to circumvent this is using a smaller radius per particle.  This 
was done in figure \ref{idealNboxR0p1}, where a radius of 0.1 was chosen.

\figOctave[htb]{idealNboxR0p5}{Ideal number of boxes for given (small) 
number of particles.  Worldsize $50 \times 50 \times 50$, radius $0.5$.}

\figOctave[htb]{idealNboxR0p1}{Ideal number of boxes for given number of 
particles.  Worldsize $50 \times 50 \times 50$, radius $0.1$.}

Note that the ideal case of $1 \times 1 \times 1$ sized boxes continues up 
to 20\,000 particles (the plateau at $2 \times 2 \times 2$ sized boxes is 
not visible anymore in that figure).

Also note that the ideal number of boxes has changed from 10 per particle 
in the previous graph to around 7 boxes per particle for large numbers of 
particles. This is likely caused by the added memory demand for the extra 
particles and boxes and is once again more of a limitation of the hardware 
than of the algorithm. Indeed, modern computers make heavy use of caching 
data from memory on fast caches near the processor. When the working set of a 
program exceeds this available space, performance will decrease. Hence it makes 
sense to limit the amount of memory accesses when dealing with a large working 
set (a lot of particles). This translates into using less boxes.

The results for an even greater amount of particles are plotted in figure 
\ref{idealNboxR0p1-1M}. The particle number reaches 1.2 million. This would 
be completely intractable using a naive approach. Once again, a plateau is 
visible, this time when using $200 \times 200 \times 200$ boxes. Also, the 
average number of boxes per particle seems to have stabilized in the 
vicinity of 10 again.

\figOctave[htb]{idealNboxR0p1-1M}{Ideal number of boxes for given (large) 
number of particles.  Worldsize $50 \times 50 \times 50$, radius $0.1$.}


Lastly, the actual time complexity is plotted for the various number of 
particles. Each simulation was done using the ideal number of boxes 
calculated above. For the case of particles with a radius 0.5, the results 
are shown in figure \ref{linearComplexityR0p5}. It is clear that the time 
complexity has gone from quadratic for the naive approach to linear when 
using space partitioning.

\figOctave{linearComplexityR0p5}{Linear complexity with ideal number of 
boxes.  Worldsize $50 \times 50 \times 50$, radius $0.5$.}

\figOctave{linearComplexityR0p1}{Small deviation from linear complexity 
with ideal number of boxes for large amount of particles.  Worldsize $50 
\times 50 \times 50$, radius $0.1$.}


In figure \ref{linearComplexityR0p1}, this graph gets extended for more 
particles (while reducing the radius of the particles to 0.1). Note the 
deviation from the linear trend. This can once again be ascribed to the 
hardware running the algorithm. The formal derivation for the linear time 
complexity assumed a constant fraction of particles in a box, $x$. We found 
above that for a small amount of particles, the ideal $x$-value was about 1/10.  
For a large amount of particles, however, this value increased to about 1/7 for 
100\,000 particles above. Thus this $x$ depends on the number of particles and 
so the theoretically expected linear curve turns out to be slightly superlinear 
in practise.

Going to even higher particle numbers, this anomaly vanishes again, as seen in 
figure \ref{linearComplexityR0p1-1M}. Note that at the point of 1.2 
\emph{million} particles, every iteration still only takes about 1 second on 
the testing machine. For comparison, this would take 5 \emph{years} with the 
naive model of figure \ref{quadraticComplexity}.

\figOctave{linearComplexityR0p1-1M}{Stabilisation of linear complexity with 
ideal number of boxes for \emph{very} large amount of particles.  Worldsize $50 
\times 50 \times 50$, radius $0.1$.}



\subsection{Conclusion}
In theory, space partitioning allows us to reduce the time complexity of a 
many-particles simulation from $O(n^2)$ to $O(n)$.

In our example, the ideal number of boxes to maximize the number of 
iterations per second turns out to be around 10 times more than the amount of 
particles. Due to the nature of the underlying hardware, this varies slightly 
as the memory demand grows larger. This also means that the ideal complexity 
that can be achieved in practice is slightly superlinear for some intermediate 
regimes.  It must be noted, however, that this superlinear complexity is 
nowhere near as bad as the quadratic complexity of the naive implementation, 
and that it turns linear again for large numbers of particles.



