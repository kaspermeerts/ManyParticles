\section{Performance}

The performance of a naieve many-particle system was evaluated and the results 
are show in figure (\ref{quadraticComplexity}). The quadratic complexity of the 
problem in function of the number of particles is clearly visible. To 
ameliorate this result, we need to determine the ideal boxnumber. The 
performance of a system in function of the number of partitions is given in 
figure (\ref{fixedPartnum}). The performance peaks around the ideal number. We 
have determined the optimum number of boxes for a large amount of particles and 
graphed it in figure (\ref{idealNbox2}).

\figOctave[htb]{idealNbox2}{Ideal number of boxes for given number of 
particles.  Worldsize $50 \times 50 \times 50$, radius $0.5$.}
\figOctave{linearComplexity}{Linear complexity with Ideal number of boxes. Worldsize $50 \times 50 \times 50$, radius $0.5$.}
\figOctave{quadraticComplexity}{Quadratic complexity with one box. Worldsize $50 \times 50 \times 50$, radius $0.5$.}

\figOctave{fixedPartnum}{Effect of number of boxes on performance, for a fixed number of particles. Worldsize $50 \times 50 \times 50$, radius $0.1$, 10000 particles.}
