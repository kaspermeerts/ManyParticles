\section{Results}
\subsection{Physical results}

To validate the correctness of the software, we have tested some important 
physical phenomena that are described by statistical mechanics. We studied the 
Maxwell-Boltzmann distribution and Brownian motion. Other interesting phenomena 
include adsorption models, the Van der Waals corrections to the ideal gas law, 
diffusion, heat conduction, phase transitions, thin film phenomena, etc...

\subsubsection{Maxwell-Boltzmann distribution}
The Maxwell-Boltzmann speed distribution describes the probability density of 
the speed of a gas particle. The necessary temperature parameter was calculated 
from our simulation using the equipartition theorem.
$$
E_{\textrm{avg, kin}} = \left< \frac{mv^2}{2} \right> = \frac{3k_BT}{2}
$$
With $m$ the mass of each particle and $k_B$ the Boltzmann constant. Both were 
set to unity in our simulation. Thus
$$
T = \frac{\left< v^2 \right>}{3}
$$
The well known Maxwell-Boltzmann distribution reads
$$
f(v)\dif v = \left( \frac{\beta m}{2 \pi} \right)^{3/2} 4 \pi v^2
\exp\left( -\beta \frac{mv^2}{2}\right) \dif v =
\sqrt{ \frac{2}{\pi T^3} } v^2 \exp \left(-\frac{v^2}{2T}\right) \dif v
$$
with T given in function of the expectation of the squared velocity above.

The velocity of our particles got initialized as a vector of fixed length 
1, pointing in a random direction. Thus, the inital speed distribution has 
the shape of a delta distribution around the speed 1. Through collisions 
between the particles, the particles reach thermal equilibrium. Whilst 
doing so, the sharp peak in the speed distribution relaxes to approximate 
the Maxwell-Boltzmann distribution.

\figOctave{maxwell}{Speed distribution of a simulation and the expected 
theoretical distribution for a temperature of 1/3}

The resulting simulated distribution is shown in figure \ref{maxwell} 
together with the theoretical result. The parameters used were: a worldsize 
of $60 \times 60 \times 60$, 8000 particles with radius 0.2 and a 
relaxation time of 15\,000 iterations with timestep 0.05.

The temperature of the theoretical curve was obtained from the formula 
above, using that the temperature at the beginning\footnote{Note that it's 
not quite appropriate to talk about a temperature in this case. We aren't 
in thermal equilibrium yet and there is no entropy to calculate $\partial E 
/ \partial S$. Nevertheless, using the equipartitiontheorem to define the 
temperature works just fine.}
was given by $\left<v^2\right>/3$, in our case 1/3.

The results are clearly in perfect agreement with the predicted result.

\subsubsection{Brownian motion}
With a small change to the software, a single massive, huge particle was 
introduced in the system. Because of its relative size, space partitioning 
could not be used for this new particle and each smaller particle had to be 
checked for interactions with the huge particle separately. This had a 
negligible effect on performance.

\figOctave{brown}{Four trajectories of Brownian motion starting from the center 
dot. (worldsize $320 \times 320 \times 320$, 10000 particles of radius 0.5, 
radius of large particle: 10, mass: 1)}

The trajectory of the particle was saved and plotted in figure \ref{brown}.  
Brownian motion is discernible in the path.

\subsubsection{Heat conduction}

Heat conduction is another phenomenon that can be studied. No results are shown 
in this article but we propose a method how one can implement this using the 
software.

The first proposal models heat conduction in a gas. We propose to give two 
opposite walls a different temperature. When a particle collides with a 
wall, the normal component with respect to the wall gets mixed randomly 
with the momentum it would have if it had the same temperature as the wall.  

This is enough to model heat transfer from one wall to another. With a 
sufficiently dense gas, and a sufficient distance between the walls, it is 
even possible to graph a temperature profile by considering the average 
energy of the particles in function of their distance to the walls.

A second proposal models heat conduction by electrons in a metal. The 
setting is the same as the previous example, but with an additional regular 
grid of stationary particles that fills the world. These particles can 
exchange kinetic energy with the moving particles after a collision. The 
stationary particle stores its kinetic energy in a vibration mode. This 
gives an idea of the temperature distribution of the system and doesn't 
require a very dense gas.
