\section{Results}
\subsection{Physical results}

To validate the correctness of the software, we have tested some important 
physical phenomena that are described by statiscal mechanics. We studied the 
Maxwell-Boltzmann distribution and Brownian motion. Other interesting phenomena 
include adsorption models, the Van der Waals corrections to the ideal gas law, 
diffusion, heat conduction, phase transitions, thin film phenomena, etc...

\subsubsection{Maxwell-Boltzmann distribution}
The Maxwell-Boltzmann speed distribution describes the probability density of 
the speed of a gas particle. The necessary temperature parameter was calculated 
from our simulation using the equipartition theorem.
$$
E_{\textrm{avg, kin}} = \left< \frac{mv^2}{2} \right> = \frac{3k_BT}{2}
$$
With $m$ the mass of each particle and $k_B$ the Boltzmann constant. Both were 
set to unity in our simulation. Thus
$$
T = \frac{\left< v^2 \right>}{3}
$$
The Maxwell-Boltzmann distribution follows
$$
f(v)\dif v = \left( \frac{\beta m}{2 \pi} \right)^{3/2} 4 \pi v^2
\exp\left( -\beta \frac{mv^2}{2}\right) \dif v =
\sqrt{ \frac{2}{\pi T^3} } v^2 \exp \left(-\frac{v^2}{2T}\right) \dif v
$$

The resulting simulated distribution is shown in figure (\ref{maxwell}) 
together with the theoretical result. The results are clearly in perfect 
agreement with the predicted result.
\figOctave{maxwell}{Speed distribution of a simulation and the expected 
theoretical distribution for a temperature of 1/9}

\subsubsection{Brownian motion}
With a small change to the software, a single massive, huge particle was 
introduced in the system. Because of its relative size, space partitioning 
could not be used and thus we had to check each particle explicitly. 

The trajectory of the particle was saved and plotted in figure (XXXBROWNXXX).  
Brownian motion is discernible in the path.


