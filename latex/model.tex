\section{Scientific model}
\subsection{Collisions}

The most basic short-range interaction that can be modeled in a many-particle 
system is the elastic collision between hard spheres (3D) or disks (2D). It is 
important that this interaction is not only addressed correctly, but also 
efficiently. Therefore, one wants to avoid calculating computationally 
expensive trigonometric functions like sine and cosine and try to keep square 
roots to a minimum.

Collision simulations function in basically two ways, the collisions can be 
detected before they happen, \emph{a priori}, or afterwards, \emph{a 
posteriori}.

In the a posteriori case, the simulation gets advanced by a small time step 
each iteration. If a collision took place, the simulation gets rolled back a 
short while to the exact moment of impact. The collision then gets handled and 
the situation is restored correctly. In the a priori case, the collision 
detection algorithm needs to interpolate the positions of various bodies to 
predict whether a collision will happen or not.

A priori algorithms are more stable and correct, because the intersections of 
bodies that need to be fixed in the a posteriori case are unphysical. However, 
determining ahead of time when two bodies will collide is very difficult and it 
requires the collision detection algorithm to know about the physics of 
moving bodies. Therefore we opted for the other method.

\subsubsection{Rewinding time}
Since we detect collisions a posteriori, we need to restore the situation to 
the exact time the collision took place. This entails moving each pair of 
intersecting sphere back in time. Assuming there is no external forcefield, the 
particles are in a state of uniform rectilinear motion. The equation we need to 
solve for the time $t$ is
$$
|\vec{r_1}(t) - \vec{r_2}(t)| = R_1 + R_2
$$
with $\vec{r_i}$ the position vectors of the bodies and $R_i$ their radii.  
Since their motion is uniform, this becomes
$$
|\vec{r_1}{t_0} + \vec{v_1}\Delta t - \vec{r_2}{t_0} - \vec{v_2}\Delta t| = | 
\Delta\vec{r} + \Delta\vec{v} \Delta t | = R_1 + R_2
$$
with $\Delta\vec{r}$ and $\Delta\vec{v}$ the particles' relative position and 
motion and $\Delta t$ the sought-after time difference. Squaring both sides, we 
find
$$
|\Delta\vec{r}|^2 + 2\Delta\vec{v} \cdot \Delta\vec{r}\Delta t + 
|\Delta\vec{v}|^2(\Delta t)^2 = (R_1 + R_2)^2
$$
The solutions of this quadratic equation are
$$
\Delta t = \frac{-2\Delta\vec{v} \cdot \Delta\vec{r}
\pm \sqrt{4( \Delta\vec{v} \cdot \Delta\vec{r})^2 -
4|\Delta\vec{v}|^2 \left( |\Delta\vec{r}|^2 - (R_1 + R_2)^2 \right) }}
{2|\Delta\vec{v}|^2}
$$
As $\Delta\vec{v} \cdot \Delta\vec{r}$ is typically a negative number and 
$R_1 + R_2$ is always greater than $|\Delta\vec{r}|$, we take the positive 
square root and simplify
$$
\Delta t=\frac{-\Delta\vec{v} \cdot \Delta\vec{r} +
\sqrt{(\Delta\vec{v} \cdot \Delta\vec{r})^2 + |\Delta\vec{v}|^2
((R_1+R_2)^2 -|\Delta\vec{r}|^2)}}
{|\Delta\vec{v}|^2}
$$
With this value, we can put the particles back in their original position 
before the collision. Now, we need to handle the collision itself.

\subsubsection{One-dimensional elastic collision}

Consider two particles denoted by subscripts 1 and 2.  Let $m_i$ be the masses, 
$v_c$ the the velocity of the center of mass (COM) frame, $v_i$ the velocities 
before the collision, $v'_i$ the velocities before the collision in the COM 
frame, $w_i$ the velocities after the collision and $w'_i$ the velocities after 
the collision in the COM frame.  We assume the speed of both bodies is 
non-relativistic. 

To considerably simplify the equations, we are going to transform to the 
inertial frame of the center of mass (COM). The COM velocity is
$$
v_c = \frac{m_1 v_1 + m_2 v_2}{m_1 + m_2}
$$

The resulting velocities in the new frame are
$$
v'_1 = v_1 - v_c = \frac{m_2}{m_1+m_2}(v_1-v_2)
$$
and
$$
v'_2 = v_2 - v_c = \frac{m_1}{m_1+m_2}(v_2-v_1)
$$

In this frame, the total momentum must be zero before and after the collision.  
The kinetic energy must also be conserved.
Thus
$$
p_1 + p_2 = q_1 + q_2 = 0
$$
and
$$
\frac{p_1^2}{2m_1} + \frac{p_2^2}{2m_2} = \frac{q_1^2}{2m_1} + 
\frac{q_2^2}{2m_2}
$$
where $p_i$ and $q_i$ are the momenta respectively before and after the 
collision.

From this follows that
$$
q_2 = -q_1 \qquad \textrm{and} \qquad p_2 = -p_1
$$
which we can fill in in the energy equation
$$
p_1^2 = q_1^2 \qquad \textrm{and} \qquad p_2^2 = q_2^2
$$
The only possible solutions are $q_i = p_i$, that is no collision happened at 
all, or $q_i = -p_i$. This means that the velocities of both bodies are 
flipped.

$$
w'_i = -v'_i
$$

Thus we can derive the new velocities in the original frame of reference
$$
w_1 = w'_1 + v_c = \frac{v_1(m_1-m_2) + 2m_2u_2}{m_1+m_2}
$$

$$
w_2 = w'_2 + v_c = \frac{v_2(m_2-m_1) + 2m_1u_1}{m_1+m_2}
$$

\subsubsection{Two- and higher-dimensional elastic collisions}
Because the colliding bodies are spheres, the only forces they can exert on 
each other at the contact point are according to the normal vector of each 
sphere at the contact point. The tangential component(s) of the momentum of 
each body do not change while the normal component is transformed according to 
the equations of an elastic collision. To get the normal vector, we simply 
normalise the displacement vector between the bodies
$$
\vec{n} = \frac{\Delta\vec{r}}{|\Delta\vec{r}|}
$$
The normal component of the velocity is
$$
v_n = \vec{v} \cdot \vec{n}
$$
The change of this component can be calculated by plugging it into the 
equations above.

\subsection{Spatial partitioning}

Most many-particle simulations incorporate some sort of interaction between the 
particles dependent on their mutual distance. Of considerable interest are 
short-range interactions, for example when the particles represent atoms. Since 
a particle could hypothetically interact with every other particle, each pair 
needs to be tested separately. Thus $n(n-1)/2$ interactions will be considered, 
with $n$ the number of particles. Most of these are superfluous because the 
particles are too far apart. This $O(n^2)$ complexity is undesirable for 
performance reasons.  An often used technique is called \textbf{spatial 
partitioning}.

By splitting the world into smaller partitions, \emph{boxes}, one only has to 
check for interactions between particles in nearby partitions. If we keep the 
particles per box a constant ($x$), the amount of boxes becomes $n/x$. The 
complexity of this problem reduces in this way to $O(x^2 \cdot n/x) = O(n)$, 
linear in the number of particles.
